\input{preambule_article.tex}

\begin{document}
\input{title.tex}
\newpage 

\tableofcontents

\newpage
\section{Введение}
 Идея проекта заключается в том, чтобы написать бота, который бы упрощал работу с учащимися в телеграмме. В этом проекте будет использован \textbf{python}, модуль \textbf{aiogram} и другое. Будет использована база данных (\textbf{sqlite3}) для хранения информации о пользователях и админах.  Бот будет разделен на 3 части:
 \begin{enumerate}
 \item[•] Админская чать
 \item[•] Клиентская чать
 \item[•] Работа с другими ботами по управлению телеграмм-каналами
 \end{enumerate}
 \section{Действия перед началом}
 Стоит это идею обсудить с преподавателями физической культуры. Также проблема в том, что у каждого преподавателя есть свой телеграмм-канал и им будет неудобно совместно управлять каналом.
 
  Составление плана интерфейса для учащихся и администраторов. Изучение функционала самого телеграма (я им почти не пользуюсь). Также \textbf{найти других участников для проекта}.
  \section{Основной функционал}
*/ Самое главное, что бот должен уметь - это создавать посты, под которыми будут отмечатся учащиеся. Текст поста примерно такой:  \textbf{17.05 пятница 18.02.2022}. В коментариях учащиеся пишут следующее:\textbf{Иванов Иван Б19-021}. После чего, в конце дня, бот должен сделать \textbf{XLS} таблицу и отправить преподавателю, ведущему в этот день занятия.  */

	Самое главное, что бот должен уметь - это создавать посты, под которыми будут отмечатся учащиеся. Пост будет сделан в виде голосование в котором чтобы принять участие студент должен отметить свою геопозицию и время (возможно телефон). Это сделано чтобы можно было проверить что студент действительно был на занятии.
	
 \section{Клиентская часть}
 Здесь я буду перечислять то, что должно быть в клавиатуре клиента (под клавиатурой будем понимать кнопки с командами). Если проекту будет дан зеленый свет, то будет создана схема команд и их зависимости.
  \begin{enumerate}
 \item[•] \textbf{Старт}(start)- приветствие нового пользователя
 \item[•] \textbf{Помощь}(help)- перечисление основных команд
 \item[•] \textbf{Преподаватель сегодня}
 \item[•] \textbf{Спортивные мероприятия}
 \item[•] \textbf{Отключить звук рассылки}
 \item[•] \textbf{Смена языка}(вряд ли сделаю)
 \item[•] \textbf{Количество посещений(отмеченные ботом)}
 \item[•] \textbf{Показать геолокацию занятий}
 \item[•] \textbf{Спортивные новости}
 \item[•] \textbf{Назначить оповещение занятий}
  \end{enumerate}
 + с тектом будут смайлики
  \section{Админская часть}
  В этой части будет функционал, доступный только админам телеграмм-каналов.
    \begin{enumerate}
   \item[•] \textbf{Рассылка сообщения}
    \item[•] \textbf{Создать пост для телеграм-канала}
    \item[•] \textbf{Создание поста о спортивных мероприятиях}
    \item[•] \textbf{Удаления поста о спортивных мероприятиях}
    \item[•] \textbf{Создание поста о спортивных новостях}
    \item[•] \textbf{Удаление поста о спортивных новостях}
    \item[•] \textbf{Создать голосование}
    \item[•] \textbf{Создать объект на карте(геоданные)}
    \item[•] \textbf{Удалить объект на карте(геоданные)}
    \item[•] \textbf{Включить бота)}
    \item[•] \textbf{Отключить бота}
    
	 \end{enumerate}
	  + с тектом будут смайлики
  \section{Работа с другими ботами}
  В телеграмме есть уже созданные беспланые боты для облегчения работы с каналами. Здесь я хочу, чтобы один бот управлял другими.
    \section{Работа с базой данных}
   Стоит заранее продумать о базе данных, чтобы можно было реализовать все клиентские функции. Будет большой проблемой, если бот не ведет учет какого-то нужного параметра клиента( Что-то исправлять будет черезвыйчано сложно ). Также стоит реализовать перевод из \textbf{SQL} в \textbf{XLS}.
     \begin{enumerate}
     \item[•] Создать БЗ для хранения id отправленных и принятых файлов
     	 \end{enumerate}
   \section{Вывод}
	Надеюсь, данный проект будет являться полезным для студентов МФТИ. Возможно, если получится его довести до логического конца, то можно будет сделать подобных ботов и для разных специальностей.   
   Некоторые вещи я уже умею делать, а некоторые даже не представляю как сделать.
\end{document}